% Вказуємо клас документа, розмір шрифту та формат паперу
\documentclass[14pt,a4paper,twoside]{article}

% --- Шрифти ---
\usepackage{lmodern} % Сучасні шрифти для кращої сумісності та масштабування

% --- Кодування та Локалізація ---
\usepackage[utf8]{inputenc} % Встановлює кодування вводу як UTF-8
\usepackage[T2A]{fontenc}   % Встановлює кодування шрифту для кирилиці
\usepackage[english,ukrainian]{babel} % Підтримка англійської та української мов

% --- Графіка ---
\usepackage{graphicx} % Для вставки графічних зображень (jpg, png, etc.)

% --- Геометрія сторінки ---
\usepackage[left=25.4mm, right=12.7mm, top=20mm, bottom=20mm]{geometry} % Встановлює поля документа

% --- Відступи та форматування тексту ---
\usepackage{indentfirst} % Додає відступ для першого абзацу після заголовка розділу

% --- Математичні пакети ---
\usepackage{amsmath}  % Додаткові математичні символи та формули
\usepackage{amsfonts} % Додаткові математичні шрифти (наприклад, \mathbb)
\usepackage{amssymb}  % Додаткові математичні символи (наприклад, \nexists, \varnothing)

% --- Таблиці ---
\usepackage{array}    % Для додаткових опцій форматування таблиць

% --- Форматування заголовків ---
\usepackage{titlesec} % Для кастомізації формату заголовків (розділів, підрозділів, тощо)

% --- Заголовки та підвали сторінок ---
\usepackage{fancyhdr} % Для кастомізації заголовків та підвалів сторінок

% --- Міжрядкові інтервали ---
\usepackage{setspace} % Для керування міжрядковими інтервалами

% --- Додаткові утиліти (Lorem Ipsum, посилання на розділи, тощо) ---
\usepackage{lipsum}   % Для генерації "заповнювача тексту" (Lorem Ipsum)
\usepackage{nameref}  % Для створення посилань на назви розділів у тексті

% --- Гіперпосилання (зазвичай підключається останнім) ---
\usepackage{hyperref} % Для створення активних гіперпосилань у PDF

% --- Налаштування формату заголовків та інших елементів ---
\titleformat{\section}[block]{\normalfont\Large\bfseries}{\thesection}{1em}{}
\titleformat{\subsection}[block]{\normalfont\large\bfseries}{\thesubsection}{1em}{}
\titleformat{\subsubsection}[block]{\normalfont\normalsize\bfseries}{\thesubsubsection}{1em}{}

% --- Відступи після заголовків ---
\titlespacing*{\paragraph}{0pt}{\baselineskip}{1em}
\titlespacing*{\subparagraph}{0pt}{\baselineskip}{1em}

% --- Налаштування сторінок ---
\pagestyle{fancy}
\fancyhf{}
\fancyhead[RO]{\thepage}
\fancyhead[LE]{\thepage}
\fancyhead[CE,CO]{}
\fancypagestyle{plain}{
	\fancyhf{}
}


\begin{document}
	
	\section{Вступ}
	Нехай буде теорема Тейлора (див. \cite{weisstein})
	\begin{theorem}\label{ttt1}
		\textbf{Теорема Тейлора.} Нехай $n\geq1$ є цілим числом, і функція $f(x)$ є $n+1$ разів диференційованою в околі точки $a\in\mathbb{R}$. Нехай $x$ є будь-яким аргументом функції з такого околу, $p$ - деяке додатне число. Тоді існує деяке $c$ між точками $a$ та $x$, таке що
		\begin{equation}\label{mvm}
			f(x)=f(a)+\frac{f'(a)}{1!}(x-a)+\frac{f''(a)}{2!}(x-a)^2+\cdots+\frac{f^{(n)}(a)}{n!}(x-a)^n+R_{n+1}(x)
		\end{equation}
		де $R_{n+1}(x)$ - загальна форма залишкового члену
		
		\begin{equation}\label{xcxxxs}
			R_{n+1}(x)=\left(\frac{x-a}{x-a}\right)^p \frac{(x-c)^{n+1}}{n!p}f^{(n+1)}(c)
		\end{equation}

		\end{theorem}
	\section{Розклад функцій у ряд Тейлора}
	
	\subsection{Розклад функції \( \sin(x) \)}
	
	\begin{definition}
		Ряд Тейлора для функції \( f(x) \) в точці \( a \) визначається наступним чином:
		\[
		f(x) = \sum_{n=0}^{\infty} \frac{f^{(n)}(a)}{n!}(x-a)^n.
		\]
	\end{definition}
	
	\begin{theorem}
		Розклад функції \( \sin(x) \) у ряд Тейлора в точці \( x = 0 \) має наступний вигляд:
		\begin{equation} \label{eq:sin_series}
			\sin(x) = \sum_{k=0}^{\infty} \frac{(-1)^k \cdot x^{2k+1}}{(2k + 1)!}.
		\end{equation}
	\end{theorem}
	
	\begin{definition}
		Позначимо \( k \)-ий член цього ряду як \( q_k \):
		\begin{equation} \label{eq:qk_def}
			q_k = \frac{(-1)^k \cdot x^{2k+1}}{(2k + 1)!}.
		\end{equation}
	\end{definition}
	
	\subsection{Рекурентні вирази}
	
	\begin{definition}
		Рекурентний множник \( R \) визначається як відношення \( (n+1) \)-го члену ряду до \( n \)-го члену ряду:
		\begin{equation} \label{eq:R_def}
			R = \frac{q_{n+1}}{q_n}.
		\end{equation}
	\end{definition}
	
	\begin{definition}
		Вираз для \( q_{n+1} \) має наступний вигляд:
		\begin{equation} \label{eq:qn1_def}
			q_{n+1} = \frac{(-1)^{n+1} \cdot x^{2(n+1)+1}}{(2(n+1) + 1)!}.
		\end{equation}
	\end{definition}
	
	\begin{theorem} \label{thm:ratio_test_result}
		Для деякого ряду, елементи якого задається виразом \(q_n = (-1)^n \cdot \frac{x^{2n+1}}{(2n + 1)!}\), відношення \(R\) між послідовними членами ряду визначається як:
		\[
		R = \frac{q_{n+1}}{q_n} = -\frac{x^2}{(2n + 2)(2n + 3)}.
		\]
	\end{theorem}
	
	\begin{proof}
		Підставляючи \eqref{eq:qn1_def} та \eqref{eq:qk_def} у \eqref{eq:R_def}, отримуємо:
		\begin{align*}
			R = \frac{q_{n+1}}{q_n} &= \frac{(-1)^{n+1} \cdot x^{2(n+1)+1}}{(2(n+1) + 1)!} \cdot \frac{(2n + 1)!}{(-1)^n \cdot x^{2n+1}}= \\
			&= -\frac{x^2}{(2n + 2)(2n + 3)},
		\end{align*}
		отже:
		\begin{equation} \label{eq:R_simplified}
			R = -\frac{x^2}{(2n + 2)(2n + 3)}.
		\end{equation}
	\end{proof}
	
	\begin{corollary}
		Знаючи \( R \) і \( q_n \), можна знайти \( q_{n+1} \) як \( q_{n+1} = R \cdot q_n \).
	\end{corollary}
	
	
	
    \begin{thebibliography}{9}

	\bibitem{weisstein} \label{weisstein}
	Weisstein, Eric W. \href{https://mathworld.wolfram.com/TaylorsTheorem.html}{"Taylors Theorem."} From MathWorld--A Wolfram Web Resource.
	\end{thebibliography}
	
	\end{document}
	
