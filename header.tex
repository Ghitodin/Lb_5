% Вказуємо клас документа, розмір шрифту та формат паперу
\documentclass[14pt,a4paper,twoside]{extarticle} % Використовуємо клас extarticle для підтримки 14pt

\usepackage[14pt]{extsizes} % Задаємо розмір шрифту 14pt

% --- Кодування та Локалізація ---
\usepackage[utf8]{inputenc} % Встановлює кодування вводу як UTF-8
\usepackage[T2A]{fontenc}   % Встановлює кодування шрифту для кирилиці
\usepackage[english,ukrainian]{babel} % Підтримка англійської та української мов

% --- Графіка ---
\usepackage{graphicx} % Для вставки графічних зображень (jpg, png, etc.)

% --- Геометрія сторінки ---
\usepackage[left=25.4mm, right=12.7mm, top=20mm, bottom=20mm]{geometry} % Встановлює поля документа

% --- Відступи та форматування тексту ---
\usepackage{indentfirst} % Додає відступ для першого абзацу після заголовка розділу

% --- Математичні пакети ---
\usepackage{amsmath}  % Додаткові математичні символи та формули
\usepackage{amsfonts} % Додаткові математичні шрифти (наприклад, \mathbb)
\usepackage{amssymb}  % Додаткові математичні символи (наприклад, \nexists, \varnothing)
\usepackage{amsthm}   % Пакет для визначення нових теоремних середовищ

% Означення стилів
\theoremstyle{definition}
\newtheorem{definition}{Означення}[section] % Означення нумеруються в межах розділу

\theoremstyle{plain}
\newtheorem{theorem}{Теорема}[section] % Теореми нумеруються в межах розділу

\theoremstyle{plain}
\newtheorem{corollary}{Наслідок}[theorem] % Наслідки нумеруються разом з теоремами

\usepackage[backend=biber,style=numeric]{biblatex}


% --- Таблиці ---
\usepackage{array}    % Для додаткових опцій форматування таблиць

% --- Форматування заголовків ---
\usepackage{titlesec} % Для кастомізації формату заголовків (розділів, підрозділів, тощо)

% --- Заголовки та підвали сторінок ---
\usepackage{fancyhdr} % Для кастомізації заголовків та підвалів сторінок

% --- Міжрядкові інтервали ---
\usepackage{setspace} % Для керування міжрядковими інтервалами

% --- Додаткові утиліти (Lorem Ipsum, посилання на розділи, тощо) ---
\usepackage{lipsum}   % Для генерації "заповнювача тексту" (Lorem Ipsum)
\usepackage{nameref}  % Для створення посилань на назви розділів у тексті

% --- Гіперпосилання (зазвичай підключається останнім) ---
\usepackage{hyperref} % Для створення активних гіперпосилань у PDF

% --- Налаштування формату заголовків та інших елементів ---
\titleformat{\section}[block]{\normalfont\Large\bfseries}{\thesection}{1em}{}
\titleformat{\subsection}[block]{\normalfont\large\bfseries}{\thesubsection}{1em}{}
\titleformat{\subsubsection}[block]{\normalfont\normalsize\bfseries}{\thesubsubsection}{1em}{}

% --- Відступи після заголовків ---
\titlespacing*{\paragraph}{0pt}{\baselineskip}{1em}
\titlespacing*{\subparagraph}{0pt}{\baselineskip}{1em}

% --- Налаштування сторінок ---
\pagestyle{fancy}
\fancyhf{}
\fancyhead[RO]{\thepage}
\fancyhead[LE]{\thepage}
\fancyhead[CE,CO]{}
\fancypagestyle{plain}{
	\fancyhf{}
}